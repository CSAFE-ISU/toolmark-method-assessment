% !TeX program = pdfLaTeX
\documentclass[12pt]{article}
\usepackage{amsmath}
\usepackage{amsthm}
\usepackage{graphicx,psfrag,epsf}
\usepackage{enumerate}
\usepackage{natbib}
\usepackage{booktabs}
\usepackage{longtable}
\usepackage{array}
\usepackage{multirow}
\usepackage[table]{xcolor}
\usepackage{wrapfig}
\usepackage{float}
\usepackage{colortbl}
\usepackage{hyperref}
\usepackage{pdflscape}
\usepackage{tabu}
\usepackage{threeparttable}

\usepackage{url} % not crucial - just used below for the URL

%\pdfminorversion=4
% NOTE: To produce blinded version, replace "0" with "1" below.
\newcommand{\blind}{0}

% DON'T change margins - should be 1 inch all around.
\addtolength{\oddsidemargin}{-.5in}%
\addtolength{\evensidemargin}{-.5in}%
\addtolength{\textwidth}{1in}%
\addtolength{\textheight}{1.3in}%
\addtolength{\topmargin}{-.8in}%

\newenvironment{definition}[1]% environment name 
{% begin code 
  \par\vspace{.75\baselineskip}\noindent 
  \textbf{Definition (#1)}\begin{itshape}% 
  \par\vspace{.5\baselineskip}\noindent\ignorespaces 
}% 
{% end code 
  \end{itshape}\ignorespacesafterend 
}

\providecommand{\tightlist}{%
  \setlength{\itemsep}{0pt}\setlength{\parskip}{0pt}}

\begin{document}

\def\spacingset#1{\renewcommand{\baselinestretch}%
{#1}\small\normalsize} \spacingset{1}


%%%%%%%%%%%%%%%%%%%%%%%%%%%%%%%%%%%%%%%%%%%%%%%%%%%%%%%%%%%%%%%%%%%%%%%%%%%%%%

\if0\blind
{
  \title{\bf Adaption of the Chumbley Score to matching of bullet striation marks}

  \author{
        Ganesh Krishnan \thanks{The authors gratefully acknowledge \ldots{}} \\
    Department of Statistics, Iowa State University\\
     and \\     Heike Hofmann \\
    Department of Statistics and CSAFE, Iowa State University\\
      }
  \maketitle
} \fi

\if1\blind
{
  \bigskip
  \bigskip
  \bigskip
  \begin{center}
    {\LARGE\bf Adaption of the Chumbley Score to matching of bullet striation marks}
  \end{center}
  \medskip
} \fi

\bigskip
% \begin{abstract}
% 
% \end{abstract}

\noindent%
{\it Keywords:} Bullets, Signatures, Lands, Errors, Known match, Known non-match,
Comparisons
\vfill

\newpage
\spacingset{1.45} % DON'T change the spacing!

\newcommand{\hh}[1]{{\textcolor{orange}{#1}}}
\newcommand{\gk}[1]{{\textcolor{green}{#1}}}
\newcommand{\cited}[1]{{\textcolor{red}{#1}}}

\section{Introduction}\label{introduction}

Compairing pairs of toolmarks with the intention of matching it to a
tool has been studied many times in the past. Extensive examples in
literature for tools and toolmark research ranging from screwdrivers to
groove pliers to slip-joint pliers can be found in the work of
\citet{manytoolmarks1}, \citet{manytoolmarks2}, \citet{miller},
\cite{afte-chumbley} and many more. In comparison to this, same source
matching of bullets to firearms has not been examined as prominently as
that of toolmarks, with even less information available on validity of
methods and error rates associated with firearms examination. The
National Academy of Sciences in its report in 2009 \citep{NAS:2009}
discussed the need for determining error rates in methods proposed for
firearms examination. In the context of same source matching and error
rates determination, as seen in the case of most forensic applications,
the first step involves identification of unique features that are
characteristic of the object at hand. For the case of bullets and
firearms, striation marks on the surface of the bullet are considered to
be such markings that can be used in methods for same source matching.
These marks are often a product of rifling and impurities and defects
due to manufacturing in the barrel of the gun, which leads to engravings
on the bullet surface \citep{afte-article1992}. In current practice,
firearm examiners invariably make visual comparisons of bullet striae
and use visual assesment tools to dignify bullets as being matches and
non-matches. One way of accomplishing anykind of comparison between
bullets is to do a comparison between surface marking of two or more
bullet lands. Bullet Lands are considered to be areas between grooves
made by the rifling action of the barrel. These marking are considered
to be unique. The land engraved markings or sometimes termed as Bullet
profiles \citep{aoas} are striation marks made on Bullet lands and often
used for these land to land comparisons. Bullet Signatures is another
word used in literature as seen in the work of \citet{chu2013} and
\citet{aoas}. In our context bullet signatures refer to a processed
version of the raw land engraved markings or profiles. The generation of
bullet signatures involves first extraction of a bullet profile by
taking the cross-sectional of the surface at a given height and then
using loess fits to model the structure. The residuals of this fit are
called signatures, which are considered to be noise free and a good
reflection of the class charecteristics and unique features of a bullet.
A more detailed version of the extraction technique of signatures is
discussed by \citet{aoas}, where comprehensive details about the height
at which profile is to be selected, removing curvature, smoothing,
identifying groove locations are explained.

In the study conducted by \citet{aoas} a machine learning based
algorithm was developed for same source matching of bullets and error
rates were discussed using the database from the Hamby Study
\citep{hamby}. In this paper, we first try to adapt a deterministic
algorithm and method developed for toolmarks by \citet{chumbley} and
improved by \citet{hadler}, to bullets. Then we consequently discuss
about the efforts in doing so along with the associated error rates. The
data used in this paper also belongs to the Hamby Study \citep{hamby}.
This gives us a common platform for comparing the performance of the
chumbley method on bullets with an already existing method proposed by
\citet{aoas} for bullets. The proposed algorithm and method of
\citet{chumbley}, in their paper, compares two toolmarks with the
intention of determining if it comes from the same source (same tool).
The method also provides a means to determine error rates and claims to
reduce subject bias. As mentioned earlier, subject bias and error rate
determination have been a long standing issue in firearm examination
\citep{NAS:2009}. This remains one of the motivations to explore the
adaptibility of the Chumbley score methodology to bullets.
\citet{chumbley} used an empirical based setup to validate their
proposed algorithm and quantitative method which calculates a U-statisic
for the purpose of classification of toolmarks as matching or
non-matching. The data for their study was obtained from 50 sequentially
manufactured screwdriver tips, and preselected comparison window sizes
were given as inputs to the algorithm. The algorithm then compares the
two toolmarks and comes up with a U-statistic and an associated p-value
to designate them as matches or non-matches. The performance for every
100 comparisons, of the algorithm proposed by \citet{chumbley} and the
improvement proposed by \citet{hadler} are listed in the table below.

\begin{table}[!htb]
    \begin{minipage}{.5\linewidth}
      \caption{Chumbley et al. 2010}
      \centering 
\begin{tabular}{lrr}
\toprule
Classification & Match & Non-Match\\
\midrule
Match & 41 & 9\\
Non-Match & 2 & 48\\
\bottomrule
\end{tabular} \end{minipage}%
    \begin{minipage}{.5\linewidth}
      \centering
        \caption{Hadler et. al. (2017)} 
\begin{tabular}{lrr}
\toprule
Classification & Match & Non-Match\\
\midrule
Match & 47 & 3\\
Non-Match & 0 & 50\\
\bottomrule
\end{tabular} \end{minipage} 
\end{table}\newpage

\bibliographystyle{agsm}
\bibliography{bibliography}

\end{document}
