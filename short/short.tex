% !TeX program = pdfLaTeX
\documentclass[12pt]{article}
\usepackage{amsmath}
\usepackage{amsthm}
\usepackage{graphicx,psfrag,epsf}
\usepackage{enumerate}
\usepackage{natbib}
\usepackage{booktabs}
\usepackage{longtable}
\usepackage{array}
\usepackage{multirow}
\usepackage[table]{xcolor}
\usepackage{wrapfig}
\usepackage{float}
\usepackage{colortbl}
\usepackage{hyperref}
\usepackage{pdflscape}
\usepackage{tabu}
\usepackage{threeparttable}

\usepackage{url} % not crucial - just used below for the URL

%\pdfminorversion=4
% NOTE: To produce blinded version, replace "0" with "1" below.
\newcommand{\blind}{0}

% DON'T change margins - should be 1 inch all around.
\addtolength{\oddsidemargin}{-.5in}%
\addtolength{\evensidemargin}{-.5in}%
\addtolength{\textwidth}{1in}%
\addtolength{\textheight}{1.3in}%
\addtolength{\topmargin}{-.8in}%

\newenvironment{definition}[1]% environment name 
{% begin code 
  \par\vspace{.75\baselineskip}\noindent 
  \textbf{Definition (#1)}\begin{itshape}% 
  \par\vspace{.5\baselineskip}\noindent\ignorespaces 
}% 
{% end code 
  \end{itshape}\ignorespacesafterend 
}

\providecommand{\tightlist}{%
  \setlength{\itemsep}{0pt}\setlength{\parskip}{0pt}}

\begin{document}

\def\spacingset#1{\renewcommand{\baselinestretch}%
{#1}\small\normalsize} \spacingset{1}


%%%%%%%%%%%%%%%%%%%%%%%%%%%%%%%%%%%%%%%%%%%%%%%%%%%%%%%%%%%%%%%%%%%%%%%%%%%%%%

\if0\blind
{
  \title{\bf Adaption of the Chumbley Score to matching of bullet striation marks}

  \author{
        Ganesh Krishnan \thanks{The authors gratefully acknowledge \ldots{}} \\
    Department of Statistics, Iowa State University\\
     and \\     Heike Hofmann \\
    Department of Statistics and CSAFE, Iowa State University\\
      }
  \maketitle
} \fi

\if1\blind
{
  \bigskip
  \bigskip
  \bigskip
  \begin{center}
    {\LARGE\bf Adaption of the Chumbley Score to matching of bullet striation marks}
  \end{center}
  \medskip
} \fi

\bigskip
% \begin{abstract}
% 
% \end{abstract}

\noindent%
{\it Keywords:} Bullets, Signatures, Lands, Errors, Known match, Known non-match,
Comparisons
\vfill

\newpage
\spacingset{1.45} % DON'T change the spacing!

\newcommand{\hh}[1]{{\textcolor{orange}{#1}}}
\newcommand{\gk}[1]{{\textcolor{green}{#1}}}
\newcommand{\cited}[1]{{\textcolor{red}{#1}}}

\section{Introduction and Results}\label{introduction-and-results}

Compairing pairs of toolmarks with the intention of matching it to a
tool has been studied relatively more in the past as compared to
bullets, and \citet{chumbley} have described in their paper an algorithm
and a deterministic method that compares two toolmarks and come to the
conclusion if they are from the same tool or not. The method also
determines the error rates, reduces subject bias and designate the two
toolmarks as matches or non-matches with respect to a source. This
project tries to adapt the Chumbley algorithm as modified by
\citet{hadler}, to bullets which are much smaller in length, width, are
not flat and curved in the cross-sectional topography as opposed to
tools like screw driver tips which produces longer and pronounced
markings. The majority of Bullet profiles and signatures extracted by
procedures mentioned by \citet{aoas} are almost 1/4 th the size of
toolmarks as used by \citet{chumbley} or even smaller. Striations on
Bullets are made on their curved surfaces, whereas the algorithm
developed by \citet{chumbley} and \citet{hadler} has only been tested
for flatter and wider surfaces which have negligible curvature.
Therefore, using methods proposed for toolmarks may need adaptation in
order to give tangible results for bullets. Moreover, in order to to get
flat bullet signatures and remove the curvatures some kind of smoothing
needs to be applied as a pre-step. This needs further investigation as
to whether the level of smoothing does effect the working of the
algorithm on Bullets. Another important aspect of adapting the algorithm
is to find the sizes of the two comparison windows \citet{hadler} that
minimizes the associated errors. This identification is not obvious as,
if we go too small in the comparison windows, the unique features of the
trace segments are lost and seem similar, while too large sizes vastly
reduces the weight of small features that would otherwise uniquely
classify a signature and hence identify the region of agreement.

An objective analysis of signatures (pre-processed markings) and
profiles (raw markings) of bullet lands for all Hamby-44 and Hamby-252
scans pairwise land to land comparisons (a total of 85,491
comparisons)\citep{hamby} made available through the NIST ballistics
database \citep{nist} was done to identify the effects of the two
comparison windows and coarseness parameter on the error rates in as
proposed in the adjusted chumbley algorithm for toolmarks by
\citet{hadler}. The results suggested that the Nominal type I error
\(\alpha\) value shows dependency on the size of the window of
optimization and the window of validation. For a given window of
optimization the actual Type I error is comparable the nominal level
only for only a select few validation window sizes. A Test Fail, which
is the percentage of incorrect classifications, depends on whether
known-match or known non-matches has predictive value, with test results
from different sources having a much higher chance to fail. On
conducting an analysis of all known bullet lands using the adjusted
chumbley algorithm, Type II error was identified to be least bad for
window of validation 30 and window of optimization 120. In case of
unsmoothed raw marks (profiles), Type II error increases with the amount
of smoothing and least for LOWESS smoothing coarseness value about 0.25
or 0.3. In an effort to identify the level of adaptiveness of the
algorithm, comparisons were made between signatures and profiles. Their
comparison with respect to validation window size for a fixed
optimization window side suggeted that, profiles have a total error
greater than or equal to the total error of signatures for all sizes of
validation window. Profiles also fail (i.e.~incorrectly classify
known-matches and known non-matches) more number of times than
signatures, which lets us conclude that the behaviour of the algorithm
for the profiles instead of pre-processed signatures is not better.
Finally it needs to be noted that the current version of the adjusted
chumbley algorithm seems to falls short when compared to other
machine-learning based methods \citet{aoas}, and some level of
modification to the deterministic algorithm needs to be identified and
tested that would reduces the number of incorrect classifications.

\bibliographystyle{agsm}
\bibliography{bibliography}

\end{document}
